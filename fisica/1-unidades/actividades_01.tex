\documentclass[11pt]{article}

\input{../../utils/preamble_problemas}

\title{Introducción a la física}
\author{Manuel Carlevaro}
\date{Universidad de Navarra}


\begin{document}
%\maketitle

\begin{center}
\framebox[1.0\textwidth][c]{
\huge{\textsc{Introducción a la física}} 
}
\end{center} 

\begin{center}
\vspace{1em}
\Large{\textsc{Universidad de Navarra}} 
\end{center}

 \vspace{1em}

\begin{center}
\begin{tabular}{r l}
 \textbf{Tema:} & Magnitudes y sistemas de unidades. Análisis dimensional. \\
                & Proceso de medida y teoría de errores. \\
 \textbf{Profesor:} & Manuel Carlevaro \\
\end{tabular}\end{center}

\vspace{2em}

\begin{pregunta}
    Un contratista de carreteras dice que al construir la cubierta de un puente él vació \num{250} yardas de concreto. ¿A qué cree usted que se refería el contratista?
\end{pregunta}

\begin{pregunta}
    En Estados Unidos, el \textit{National Institute of Science and Technology} (NIST) mantiene varias copias exactas del kilogramo estándar internacional. A pesar de una cuidadosa limpieza, estos estándares nacionales aumentan de masa a razón de \qty{1}{\ug} /año en promedio, en comparación con el kilogramo estándar internacional. (Se comparan cada diez años aproximadamente.) ¿Es importante este cambio aparente? Explique su respuesta.
\end{pregunta}

\begin{pregunta}
El número $\pi$ no tiene unidades, ¿por qué?
\end{pregunta}

\begin{pregunta}
 Suponga que le dicen que un cilindro de radio $r$ y altura $h$ tiene un volumen dado por $\pi r 3 h$. Explique por qué esto no puede ser correcto.
\end{pregunta}

\begin{pregunta}
Una lotería ofrece 10 megaeuros como premio principal, pagaderos en un lapso de 5 años, ¿de qué monto es cada cheque semanal?
\end{pregunta}

\begin{exercise}
Exprese en la unidad requerida:
\begin{enumerate}[a)]
    \item Pasar a \unit{\hecto m}: i) \qty{3.7}{m}, ii) \qty{0.02}{\deca m}, iii) \qty{0.21}{\km}
    \item Pasar a \unit{\kg}: i) \qty{2784}{\deci g}, ii) \qty{2.05}{\deca g}, iii) \qty{435}{g}
    \item Pasar a segundos: i) \qty{30}{min}, ii) \qty{1.5}{h}, iii) \qty{0.25}{min}
    \item Pasar a horas: i) \qty{30}{min}, ii) \qty{1}{h} \qty{30}{min}, iii) \qty{5000}{s}
    \item Pasar a \unit{\deci m^3}: i) \qty{2}{m^3}, ii) \qty{3500}{cm^3}, iii) \qty{0.0025}{\kilo l}
    \item Pasar a \unit{\cm^3}: i) \qty{0.035}{m^3}, ii) \qty{750}{\deci m^3}, iii) \qty{1100}{l}
\end{enumerate}
\end{exercise}

\begin{exercise}
¿Cuáles de las siguientes expresiones indican la misma cantidad?
\begin{enumerate}[a)]
    \item \qty{2}{m} \qty{50}{cm}: \qty{250}{cm} \blank{}, \qty{205}{cm} \blank{}, \qty{25000}{\mm} \blank{}, \qty{2050}{\mm} \blank{}, \qty{2.5}{m} \blank{}.
    \item \qty{3}{kg} \qty{400}{g}: \qty{3400}{g} \blank{}, \qty{3400}{kg} \blank{}, \qty{304}{\centi g} \blank{}, \qty{3040}{\mg} \blank{}, \qty{3.4}{kg} \blank{}.
    \item \qty{5}{l} \qty{400}{ml}: \qty{540}{ml} \blank{}, \qty{5400}{ml} \blank{}, \qty{5040}{\milli l} \blank{}, \qty{5.4}{l} \blank{}, \qty{540}{\centi l} \blank{}.
    \item \qty{1}{h} \qty{45}{min}: \qty{105}{min} \blank{}, \qty{1.50}{h} \blank{}, \qty{1.75}{h} \blank{}, \qty{6300}{s} \blank{}, \qty{3600}{s} \blank{}.
\end{enumerate}
\end{exercise}

\begin{exercise}
    El volumen de una mochila es de \qty{1.5}{ft^3} . Si se necesita llevar una carga con un volumen de \num{45} litros ¿podremos poner todo en esa mochila? ¿cuántos \unit{m^3} entran en la mochila? (Recuerda $\qty{1}{ft} = \qty{0.3048}{m}$.)
\end{exercise}

\begin{exercise}
Convierta las siguientes unidades derivadas a las requeridas:
\begin{enumerate}[a)]
    \item \qty{30}{km/h} a \unit{m/s}.
    \item \qty{20}{m/s} a \unit{km/h}.
\end{enumerate}
\end{exercise}

\begin{exercise}
    La densidad del plomo es \qty{11.3}{g/cm^3} . ¿Cuál es su equivalencia en kilogramos por metro cúbico?
\end{exercise}

\begin{exercise}
    Un mol de agua (esto es \num{6,022e23} moléculas de agua) pesa \qty{18}{g}. ¿Cuántos \unit{kg} pesa una única molécula de agua? ¿Cuántos \unit{\pico g}?
\end{exercise}

\begin{exercise}
    Con una regla graduada de madera, usted determina que un lado de un trozo rectangular de lámina mide \qty{12}{mm}, y usa un micrómetro para medir el ancho del trozo, obteniendo \qty{5.98}{mm}. Conteste las siguientes preguntas con las cifras significativas correctas. 
    \begin{enumerate}[a)]
        \item ¿Qué área tiene el rectángulo? 
        \item ¿Qué razón ancho/largo tiene el rectángulo?
        \item ¿Qué perímetro tiene el rectángulo? 
        \item ¿Qué diferencia hay entre la longitud y la anchura?
    \end{enumerate}
\end{exercise}
\end{document}
